\documentclass{beamer}
\usepackage[utf8]{inputenc}
\usepackage{hyperref}
\usepackage{multicol}
\usepackage{hyperref}

\inputencoding{utf8}

\mode<presentation> {
    \usetheme{Madrid}
}

\usepackage{graphicx}
\usepackage{booktabs}

\title[Introducciòn]{Introducci\'on al curso Inform\'atica 2}
\author{Ernesto Rodriguez}
\institute{
    Universidad del Itsmo \\
    \medskip \textit{erodriguez@unis.edu.gt}
}

\date[\today]{}

\begin{document}

\begin{frame}
\titlepage
\end{frame}

\begin{frame}
    \frametitle{Objetivos del curso}
    \begin{itemize}
        \item{Reforzar y desarrollar destrezas de programaci\'on, razonamiento l\'ogico y abstracto.}
        \item{Familiarizarse con los m\'etodos m\'as avanzados para implementar ideas mediante la programaci\'on.}
        \item{Aprender principios y practicas importantes para escribir programas de alta calidad.}
        \item{Conocer algunos campos de las ciencias de la computaci\'on como teoria de conjuntos, complejidad algoritmica, matematica discreta, etc.}
        \item{Familiarizarse con herramientas y practicas que apoyan a las ciencias de la computaci\'on.}
        \item{Motivar la curiosidad, pasion e interes de los estudiantes en las ciencias de la computaci\'on y tecnologia.}
    \end{itemize}
\end{frame}

\begin{frame}
    \frametitle{Formato del curso}
    \begin{itemize}
        \item{Una hoja de trabajo (casi) semanal.}
        \item{Se utilizara Git y Github para entregar todo.}
        \item{Todo trabajo escrito se realizara con Latex.}
        \item{Tres ex\'amenes parciales t\'eoricos.}
        \item{Un proyecto (50\% del ex\'amen final.)}
        \item{Ex\'amen final, parte t\'eorica.}
        \item{Agnostico a plataforma y editor.}
        \item{No es un curso de programaci\'on, pero habra bastante de ella.}
    \end{itemize}
\end{frame}

\begin{frame}
\frametitle{Herramientas y Recursos}
\begin{multicols*}{2}
    {\bf Herramientas} \\
\begin{itemize}
    \item \href{https://code.visualstudio.com/}{Visual Studio Code}
    \item \href{https://git-scm.com/}{Git}
    \item \href{https://github.com/}{Github}
    \item \href{https://www.latex-project.org/}{Latex}
    \item \href{https://dotnet.github.io/}{.Net Core}
    \item \href{https://antergos.com/}{Antergos Linux}

\end{itemize}
\columnbreak
{\bf Recursos}
\begin{itemize}
    \item Latex Wiki \cite{Latex}
    \item Git tutorial \cite{GitTutorial}
    \item .Net Core Guide \cite{DotNetGuide}
\end{itemize}
\end{multicols*}
\end{frame}

\begin{frame}
\frametitle{Referencias}
\bibliography{../../Referencias/referencias}
\bibliographystyle{plain}
\end{frame}

\end{document}